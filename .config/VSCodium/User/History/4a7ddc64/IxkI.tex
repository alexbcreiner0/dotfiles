\documentclass[12pt]{article}
\usepackage[margin=1in]{geometry}
\usepackage{listings}
\usepackage{amsmath}
\usepackage{amsfonts}
\usepackage{fancyvrb}
\usepackage{minted}
\usepackage{graphicx}
\usepackage{enumitem}  %this permits the enumeration items to flow past sections

\title{Algorithms - Assignment 6}
\date{Due Thursday, March 28th by 11:59 pm}

\begin{document}
\maketitle

\begin{enumerate}
    \item The difference between breadth-first and depth0first search is best seen through carrying out the operation on trees. 
    \item[(1)] Perform Dijkstra's shortest paths algorithm on the following graph (graph $\alpha$) with $A$ as the starting node. Along the way you should
    \begin{itemize}
        \item[(a)] Create a table showing the intermediate distance values of all nodes after each iteration of the algorithm (i.e. there should be a column for each time a new node is released from the priority queue, and a row for every node). Make sure to include an extra column and row labelling everything (i.e. have an `A' left of the A row, a 1 above the first iteration column, etcetera), so the graders can understand what they are looking at. Alongside these distances you should also include the current best predecessor (i.e. the \emph{prev} values associated with each node). You don't need to include the state of the priority queue after each iteration, but you will almost certainly need to write it down on paper in order to carry out the algorithm.
        \item[(b)] Afterwards, draw the shortest-paths tree. 
    \end{itemize}
    \begin{figure}[h]
        \centering
        \includegraphics[width=1\textwidth]{hw6_graph1.png}
        \caption{Graph $\gamma$}
    \end{figure}
    \item[(2)] In this problem, you will use the priority queue data structure that you created in homework 4 in order to implement Dijkstra's algorithm. 
    \begin{itemize}
        \item[(a)] Before that though, we need to make a few changes/additions to that class (I tried to leave you with something which would be perfect for the job in that homework, but my planning was not perfect). Don't worry, this part shouldn't take long.
        \begin{itemize}
            \item Replace the increase\_priority(self, task, k) function with a function called change\_priority(self, task, k). Instead of adding k to the current priority, it should simply replace the old priority with k. This is better suited for Dijkstra. 
            \item The biggest addition we need to make is that our priority queue needs a min-mode, in which the lowest numbers have the highest priority. We can do this in same way that we talked about turning a max-heap into a min-heap. 
            \par In the constructor, add an additional default argument called mode, which defaults to the string 'max'. The constructor should initialize an attribute of the same name to that argument. (I.e. add the line self.mode = mode). \par 
            All we need to change in order to make everything work in min-mode is add a few checks to the enqueue(self, task, priority), dequeue(self, task, priority), and change\_priority(self, task, k). For enqueue, simply add a check which multiplies the input priority by -1 if the mode attribute is set to 'min'. This makes it so that behind the scenes, all of the priorities will be negative. But the user doesn't need to care about it. When the user dequeues something, they should have their original priority returned, not the negative priority, so add a mode check to that method which multiplies by -1 again before returning the when in min-mode. Finally, in the change\_priority method, add a mode check which multiplies k by -1 when in min-mode before doing anything with it. That should be all you need to change in order to switch your priority queue between a max mode and a min mode. 
            \item These less necessary, but will make Dijkstra a little more straightforward to implement: add two more dunder methods, \_\_len\_\_(self) and \_\_contains\_\_(self, task). The first is what python looks for when len(thing) is called. Have your queue return the size of the underlying heap when len(Q) is called (one line of code). \_\_contains\_\_(self, task). \_\_contains\_\_(self, task) is what python looks for when you write an if statement which checks membership. I.e.
            \begin{minted}{python3}
                l = [1,2,3,4]
                if 1 in l: print("Found it")
            \end{minted}
            When python sees the `in' here, it looks for a method called \_\_contains\_\_, which is supposed to return a boolean True or False. Make such a method check to see if a task is currently in the queue (you can do this in constant time using the data\_locations dictionary attribute). 
        \end{itemize}
        \item[(b)] With that out of the way, your queue should be ready for use in Dijkstra. Program it the way we discussed in class. Your function dijkstra(G,s) should take two arguments, a weighted graph G and a starting node s. Further instructions:
        \begin{itemize}
            \item The graph should be a weighted adjacency list in the form of a dictionary of dictionaries. Rather then bother explaining exactly how this should look, I've included an example graph on canvas, which is a list representing the graph from problem 1 of this assignment. You already know what the output should be, so you can use it to see if your code is working.
            \item Have the starting node $s$ default to the first node in the adjacency list. After that and before doing anything else, your function should test to see if the starting node is an actual node in the graph. If it isn't, then return an error message and stop.
            \item The function should return two outputs: dists, a dictionary of shortest distances to $s$ for each node, and prevs, a dictionary of predecessors in the shortest paths tree, so that shortest paths can be recovered. 
        \end{itemize} 
    \end{itemize}

\end{enumerate}


\end{document}