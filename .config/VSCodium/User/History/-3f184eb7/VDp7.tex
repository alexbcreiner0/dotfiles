% Maybe need something involving sets still
% Probably can't have too much Turing machine stuff if it's optional. 

\documentclass[12pt]{article}
\usepackage[utf8]{inputenc}
\usepackage[margin=0.8in]{geometry}
\usepackage{ragged2e}
\usepackage{amsmath,amssymb}
\usepackage{minted}
\usepackage{listings}
\usepackage{wrapfig}
\usepackage{tikz}
\usetikzlibrary{chains,fit,shapes}

\title{CSCI 3383 \\Final Exam}
\date{}

\begin{document}
\maketitle
\textbf{\emph{Full, printed} name: }
\vspace{1cm}

\textbf{Instructions}: 
\begin{itemize}
    \item Read the instructions
    \item Stuck? I will allow everyone \textbf{one} hint during the exam. Raise your hand and I will try to get you moving in the right direction.  
    \item You are allowed scratch paper and a calculator. 
    \item Have a wonderful holiday break!
\end{itemize} 

% Problems:
% DFS a graph
% Label pre/post nums, identify edge types
% Check for connectedness algorithm pseudocode?
% Toposort a graph
% BFS shortest paths a graph
% Dijkstra's a graph
% DP problem
% Give a DP solution, ask to modify to return the object

\newpage
\begin{enumerate}
    % Quick/conceptual/multiple choice   
    \item The following graph is:
    \begin{itemize}
        \item Connected
        \item Strongle connected
        \item Neither
    \end{itemize}
    \item When would you want to use an adjacency list over an adjacency matrix? 
    
    \item When would you want to use an adjacency matrix over an adjacency list?
    \item Let $G = (V,E,w)$ be a directed weighted graph. In general, finding shortest paths takes time \rule{1cm}{0.15mm}. However, if the graph has nonnegative edge weights, then it takes time \rule{1cm}{0.15mm}, and if it is acyclic, it takes time \rule{1cm}{0.15mm}. 
    \begin{itemize}
        \item $O((|V|+|E|)\log(|V|))$, $O(|V||E|)$, $O(|V|+|E|)$
        \item $O(|V||E|)$, $O(|V|+|E|)$, $O((|V|+|E|)\log(|V|))$
        \item $O(|V|^2)$, $O((|V|+|E|)\log(|V|))$, $O(|V|+|E|)$
        \item $O(|V||E|)$, $O((|V|+|E|)\log(|V|))$, $O(|V|+|E|)$
    \end{itemize} 
    \label{connected_multi_choice}
    \begin{figure}{h}
        \caption{Graph for problem \ref{connected_multi_choice}}
        \includegraphics[width=5cm]{graph_2.png}
    \end{figure}   
    \item Consider the following graph:
    \begin{figure}{h}
        \caption{Graph for problem }
        \includegraphics[width=10cm]{graph_1.png}
    \end{figure}
    \begin{itemize}
        \item[(a)] Write this graph as an adjacency list.
        \item[(b)] Write this graph as an adjacency matrix.
    \end{itemize}
    % Longer
    \item Consider the following graph: 
    \begin{figure}{h}
        \caption{Graph for problem \ref{dfs_prob}}
    \includegraphics[width=10cm]{dfs_graph.png}
    \begin{itemize}
        \item Perform a depth first search of the graph, and draw the resulting DFS forest. Start at node $A$, and proceed alphabetically whenever there is a choice of node to visit next. 
        \item Classify each of the non-tree edges.  
        \item Is the graph cyclic? Explain your answer in terms of part b. 
    \end{itemize}
    \end{figure}
    \label{dfs_prob}
    \item Consider the following dag:
    \begin{figure}{h}
        \caption{Graph for problem \ref{topo_sort_prob}}
        \includegraphics[width=10cm]{topo_sort_graph.png}
    \end{itemize}
    \begin{itemize}
        Topologically sort the graph using the algorithm we discussed in class. Show all work necessary for me to understand that you are correctly using the algorithm. 
    \end{itemize}
    \label{topo_sort_prob}
    \item A fast food chain is considering opening a series of restaurants along a certain highway. There are $n$ possible locations where stores can be opened along this highway. The distances of these locations from the start of the highway are, in miles and in increasing order, $m_1, m_2, \ldots, m_n$. The constraints are as follows:
    \begin{itemize}
        \item For each location, there is an expected amount of profit to be made from opening a restaurant, $p_1, p_2, \ldots, p_n$ (all of these are greater than $0$).
        \item Any two restaurants should be at least $k$ miles apart from each other, where $k$ is a positive integer. 
    \end{itemize} 
    Give a dynamic programming algorithm which will compute the maximum expected total profit subject to the given constraints. (Call this quantity $MP(n)$) Your description should clearly define subproblems, and state a mathematical recurrence relation between the problem and it's subproblems. You don't need to write any pseudocode, but it should be clear to me how one would go from what you've written to a working python program. 

    \item[(2)] 
\end{enumerate}
\end{document}