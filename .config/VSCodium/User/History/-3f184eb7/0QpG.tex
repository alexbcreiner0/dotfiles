% Maybe need something involving sets still
% Probably can't have too much Turing machine stuff if it's optional. 

\documentclass[12pt]{article}
\usepackage[utf8]{inputenc}
\usepackage[margin=0.8in]{geometry}
\usepackage{ragged2e}
\usepackage{amsmath,amssymb}
\usepackage{minted}
\usepackage{listings}
\usepackage{wrapfig}
\usepackage{tikz}
\usetikzlibrary{chains,fit,shapes}

\title{CSCI 3383 \\Final Exam}
\date{}

\begin{document}
\maketitle
\textbf{\emph{Full, printed} name: }
\vspace{1cm}

\textbf{Instructions}: 
\begin{itemize}
    \item Read the instructions
    \item Stuck? I will allow everyone \textbf{one} hint during the exam. Raise your hand and I will try to get you moving in the right direction.  
    \item You are allowed scratch paper and a calculator. 
    \item Have a wonderful holiday break!
\end{itemize} 

% Problems:
% DFS a graph
% Label pre/post nums, identify edge types
% Check for connectedness algorithm pseudocode?
% Toposort a graph
% BFS shortest paths a graph
% Dijkstra's a graph
% DP problem

\newpage
\begin{enumerate}
    % Quick/conceptual/multiple choice
    \begin{wrapfigure}{h}{5cm}
        \includegraphics[width=5cm]{graph_2.png}
    \end{wrapfigure}      
    \item The following graph is:
    \begin{itemize}
        \item Connected
        \item Strongle connected
        \item Neither
    \end{itemize}
   
    \item Consider the following graph:
    % Longer
    \item A fast food chain is considering opening a series of restaurants along a certain highway. There are $n$ possible locations where stores can be opened along this highway. The distances of these locations from the start of the highway are, in miles and in increasing order, $m_1, m_2, \ldots, m_n$. The constraints are as follows:
    \begin{itemize}
        \item For each location, there is an expected amount of profit to be made from opening a restaurant, $p_1, p_2, \ldots, p_n$ (all of these are greater than $0$).
        \item Any two restaurants should be at least $k$ miles apart from each other, where $k$ is a positive integer. 
    \end{itemize} 
    Give a dynamic programming algorithm which will compute the maximum expected total profit subject to the given constraints. (Call this quantity $MP(n)$) Your description should clearly define subproblems, and state a mathematical recurrence relation between the problem and it's subproblems. You don't need to write any pseudocode, but it should be clear to me how one would go from what you've written to a working python program. 

    \item[(2)] 
\end{enumerate}
\end{document}