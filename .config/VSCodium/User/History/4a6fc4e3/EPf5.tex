\documentclass[12pt]{article}
\usepackage[margin=1in]{geometry}
\usepackage{listings}
\usepackage{amsmath}
\usepackage{amsfonts}
\usepackage{fancyvrb}
\usepackage{minted}
\usepackage{graphicx}
\usepackage{enumitem}  %this permits the enumeration items to flow past sections

\title{Algorithms - Assignment 5}
\date{Due Thursday, April 11th by 11:59 pm}

\begin{document}
\maketitle
\begin{lstlisting}
    Here is a line of code
    Another line(x)
\end{lstlisting}
This assignment does not require any coding, and can thus be submitted as a pdf, like you did for assignment 1. The assignment expects you to draw graphs. The latex tool for this is a library called TikZ, which is a very robust and complicated tool for drawing all sorts of figures that I've frankly never had the patience to learn. There are plenty of other tools available for making graphs on a computer, and you are welcome to use any of them, but I think it would actually be better for your learning process if you drew the graphs by hand for this assignment first. You can either scan your drawings and attach them to your latex write-up as pictures, or use software to recreate the graphs and attach those pictures instead - both of these will receive the full latex bonus. 
\begin{enumerate}
\item[(1)] Consider the undirected graph $G = (V,E)$, where $V = \{A,B,C,D,E,F,G,H,I\}$ and $E = \{(A,B), (A,E), (B,C), (B,E), (C,F), (D,H), (D,G), (E,F), (F,I), (G,H)\}$.
\begin{itemize}
    \item[(a)] Draw the graph. For the sake of uniformity, make it a 3x3 grid, with A, B, and C on row 1, D, E, and F on row 2, and G, H, and I on row 3.
    \item[(b)] Represent the graph as an adjacency list. 
    \item[(c)] Represent the graph as an adjacency matrix.
    \item[(d)] Perform a depth-first search on this graph, and draw the DFS forest generated by that search. Again for uniformity, whenever there is a choice of vertices, always pick the one that comes first in the alphabet.
    \item[(e)] From the DFS, identify back edges. How many are there? 
\end{itemize} 
\item[(2)] Consider the following digraphs $\alpha$ and $\beta$, wherever they happen to show up in this document (we're all sure aware at this point of how capricious latex can be with where it puts these things):
\begin{figure}[h]
    \centering
    \begin{minipage}{0.45\textwidth}
        \centering
        \includegraphics[width=0.9\textwidth]{graph2a.png} % first figure itself
        \caption{Graph $\alpha$}
    \end{minipage}\hfill
    \begin{minipage}{0.45\textwidth}
        \centering
        \includegraphics[width=0.9\textwidth]{graph2b.png} % second figure itself
        \caption{Graph $\beta$}
    \end{minipage}
\end{figure}
\begin{itemize}
    \item[(a)] For \emph{graph $\alpha$ only}, represent it as both an adjacency list and as an adjacency matrix 
    \item[(b)] For both graphs, perform a depth-first search, drawing the DFS forest as you go. Again as before, whenever there is a choice of vertices, pick the first one alphabetically. 
    \item[(c)] For both graphs, identify the forward edges, back edges, and cross edges. This is doable by sight once you've drawn the DFS forests, but if you aren't sure how, consider doing part d first for a more mechanical approach. (Note that, obviously, for very large graphs, this is not at all doable by sight, especially not with the graphs drawn visually, hence why we care about finding algorithms for these things.)
    \item[(d)] For both graphs, label all vertices with their pre and post numbers. For graph $\alpha$ only, use the facts we derived in class relating edge types to these numbers to verify that they are classified correctly.
    \item[(e)] A confusing but important thing to understand about this edge taxonomy for digraphs is how they actually relate to the graph directly, if at all. Are these edge types properties of the graph? Or are they properties of the particular DFS we ran? For graph $\alpha$ only, perform DFS again, starting with $A$, but going backwards alphabetically instead of forwards (i.e. whenever there is a choice of vertex, choose one which comes last alphabetically). Draw the forest as you go and then identify forward, backward, and cross edges. What is different now? How does this answer the question we just posed?
\end{itemize}
\item[(3)] Suppose we have three liquid containers which can contain a maximum amount of 10 pints, 7 pints, and 4 pints, respectively. Initially, the 7-pint and 4-pint containers are full of water, and the 10-pint container is empty. We are allowed to do exactly one operation on these containers: pouring the contents of one container into another, stopping \emph{only} when either the container being poured into is empty or when the container being poured from is full. We want to know if there is a sequence of pourings that leaves exactly 2 pints in either the 7 pint or 4 pint container.
\begin{itemize}
    \item[(a)] Model this as a graph problem. Give a precise definition of the graph involved and state the specific question about this graph that needs to be answered. 
    \item[(b)] Write pseudocode for an algorithm which solves the problem.
\end{itemize}  
\item[(4)] Consider the following dag $\gamma$ wherever it shows up in this document:
\begin{figure}[h]
    \centering
    \includegraphics[width=0.7\textwidth]{graph4.png}
    \caption{Graph $\gamma$}
\end{figure}
\begin{itemize}
    \item[(a)] Perform a DFS and draw the DFS-forest, labelling pre and post numbers for each vertex. 
    \item[(b)] What are the sources of this graph? What are the sinks?
    \item[(c)] Perform a topological sort of this graph using the post numbers you found in part a. Your final answer should be the vertices in sorted order, and it should be clear to a reader how you arrived at this number. 
    \item[(d)] How many other possible topological orderings are there?
\end{itemize}
\item[(5)] Suppose that a CS curriculum consists of $n$ many courses, all of which are mandatory to graduate. Some courses have prerequisites, others don't. Some courses have prerequisites which themselves have prerequisites, and so forth. Imagine a prerequisites graph $G$ where each class has a corresponding vertex, and where there is an edge from $a$ to $b$ if $a$ is a prerequisite for $b$. Note that $G$ is clearly a dag. (Otherwise graduation would be impossible!) \par Given this graph in the form of an adjacency list, describe and write pseudocode for an algorithm which returns both the minimum number of semesters required to graduate, as well as a full schedule of which courses are to be taken each semester. Assume that there are no limits to the number of courses which can be taken each semester, and that students are in a hurry, i.e. will always take the maximum number of courses possible in a semester. Your solution should work in linear time. 
\end{enumerate}


\end{document}